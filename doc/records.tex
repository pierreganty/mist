\documentclass[nochap]{apuntes}

\title{Testing Mist}
\author{Pedro Valero}
\date{14/15 C1}

% Paquetes adicionales

% --------------------

\begin{document}
\pagestyle{plain}
\maketitle

\tableofcontents
\newpage

\section{Wednesday, 29 October 2014}

\subsection{I installed misti in my computer}
Following the steps given on the README I executed the followed commands:
\begin{enumerate}
\item \textbf{./autogent.sh}.

Executed without problems

\item \textbf{./configure}.

Executed without problems

\item \textbf{make}.

It trhows an error because the command \textit{yacc} was not found.

I installed the package \textit{bison} which includes this command and repeat this step. Now, it seems to work.

\item \textbf{make install}

Although this step is optional I did it to check if all works fine.

Throws a permision denied error.

I executed this command again with superuser privileges and it seem to work.

\textbf{Note:} For some commands it didn't do anything:
\begin{itemize}
\item install-data-am
\item install-exec-am
\item install-data-am
\end{itemize}
\end{enumerate}

\subsection{Testing program}
I wrote a python script which takes all the files in the folder given as argument with extension .spec and uses mist over them.

Right now the script seems to work but I'm not sure about how do I have to interpret the output to recognise when the targe has been reached or not.

\section{Thursday, 30 October 2014}
\subsection{Analyzing the output}
I take a look at \url{http://www.ulb.ac.be/di/ssd/ggeeraer/eec/tableau/} and I found some differences between the results given there and results obtained in my computer. For example:
\begin{itemize}
\item The example \textbf{Java} is said to be not safe in the table but when I executed the tool with it the output says:

"EEC concludes safe with the abstraction"

"backward algorithm concludes unsafe"

"TSI concludes safe"

\item The example \textbf{MOESI} is said to be safe but when I executed it in my computer I obtained:

"Cannot assign non null value to a variable in transition  5"

for any algorithm used.
\end{itemize}

\subsection{Running all tests}
After see this I modified my script to take the conclusions given by each algorithm for each test given in the folder 'examples'. After chech with some examples that the script seems to work, I let the computer working all night.

\section{Friday, 30 October 2014}
\subsection{Collecting results of the tests}
Althought the computer has been working all night it didn't end the job. To collect, at least, a first aproximation of the tests I forced to end this executions of the tools which takes too long.

The executions that I forced to end by keyboard interrupt are:
\begin{itemize}
\item \begin{verbatim} mist --tsi ../examples/PN/bingham\_h150.spec \end{verbatim}
\item \begin{verbatim} mist --backward ../examples/PN/bingham\_h250.spec \end{verbatim}
\item \begin{verbatim} mist --tsi ../examples/PN/bingham\_h250.spec \end{verbatim}
\item \begin{verbatim} mist --eec ../examples/PN/extendedread-write- \end{verbatim}smallconsts.spec
\item \begin{verbatim} mist --tsi ../examples/PN/extendedread-write- \end{verbatim}smallconsts.spec
\item \begin{verbatim} mist --eec ../examples/PN/extendedread-write.spec \end{verbatim}
\item \begin{verbatim} mist --tsi ../examples/PN/extendedread-write.spec \end{verbatim}
\item \begin{verbatim} mist --eec ../examples/PN/fms\_attic.spec \end{verbatim}
\item \begin{verbatim} mist --eec ../examples/PN/kanban.spec \end{verbatim}
\item \begin{verbatim} mist --eec ../examples/PN/mesh3x2.spec \end{verbatim}
\item \begin{verbatim} mist --tsi ../examples/PN/mesh3x2.spec \end{verbatim}
\end{itemize}

Apart from this, there are some examples whose output didn't math with the results given in \url{http://www.ulb.ac.be/di/ssd/ggeeraer/eec/tableau/} or, at least, cause some confusion.

These examples are:
\begin{enumerate}
\item There are so many examples that \textbf{throws an error}:

``Cannot assign non null value to a variable in transition x''
\begin{itemize}
\item MOESI
\item last-in-first-served
\item german-protocol
\item berkeley
\item dragon
\item futurebus
\item illinois
\end{itemize}
The problem is the same for each algorithm applied

\item Other examples causes a \textbf{segmentation fault}:
\begin{itemize}
\item fms backward
\item kanban backward
\item newrtp backward
\item firefly tsi
\end{itemize}
\item The examples 'german', 'consprod', 'delegatebuffer', 'examplelea', 'transthesis', \textbf{should be  safe} but the results obtained by the tool are:
\begin{verbatim}
``EEC concludes unsafe''
``backward algorithm concludes safe''
``TSI concludes unsafe''
\end{verbatim}

\item The examples 'simplejavaexample' and 'Java' \textbf{should be unsafe} but the results are:
\begin{verbatim}
``EEC concludes safe''
``backward algorithm concludes unsafe''
``TSI concludes safe''
\end{verbatim}

\item The example 'kanban' \textbf{should be safe }but the results are:
\begin{verbatim}
EEC keyboard interrupt (too long)
``backward algorithm concludes unsafe''
``TSI concludes unsafe''
\end{verbatim}
\end{enumerate}

\subsection{Learning somthing about Petri nets}
I started to look for information about Petri nets to get a better knodeledge about what the tool does.

I got the needed information from \url{http://en.wikipedia.org/wiki/Petri_net} and \url{http://condor.depaul.edu/rjohnson/dm6th/petri.pdf}.

After reading it I think that the tool give us some results which are inconsistent which the older ones.

Until Pierre could give me an answer about if there is a problem with the tool or I'm wrong, I'm going to tray to understand how does the tool works and what does exactly do.

\section{Monday, 3 November 2014}
\subsection{Updating python script}
After Pierre give me an explanation about how to interpret the results given by 'mist' I made a new python script to give us a structured summay about the consistence of the mist's results in relation with the results from the web.

I also collect the official results into a file to been used by the script. The file which contains these results is called ``expected\_results.txt''

\normalsize
\printindex
\end{document}